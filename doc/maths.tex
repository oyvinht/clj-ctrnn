\documentclass[]{article}
\usepackage{acronym}

\title{Mathematical background for CTRNNs}

\begin{document}

\begin{acronym}
  \acro{CTRNN}{Continuous-Time Recurrent Neural Network}
\end{acronym}

The clj-ctrnn library simulates \acs{CTRNN}s using the \textit{forward Euler method} for approximating a series of solutions to the ordinary differential equation that govern neuron membrane potentials. The equation is:
\begin{equation}
  \dot{y}_i = \frac{1}{\tau_i}(-y_i + \sum_{j=1}^{N}w_{ji}\sigma(y_j + \theta_j) + I_i)
\end{equation}
where:\\
\begin{tabular}{ll}
  $\tau_i$: & time-constant of post-synaptic neuron\\
  $y_i$: & membrane potential of post-synaptic neuron\\
  $w_{ji}$: & weight of connection between pre-synaptic neuron $j$ and post-synaptic neuron\\
  $\sigma(x)$: & the sigmoid activation function in equation \ref{sigmoid-function}\\
  $y_j$: & membrane potential of pre-synaptic neuron $j$\\
  $\theta_j$: & bias (input sensivity) of pre-synaptic neuron $j$\\
  $I_i$: & any external input (such as a sensor reading) to post-synaptic node\\
\end{tabular}
\begin{equation}
  \label{sigmoid-function}
  \sigma(x) = \frac{1}{1 + e^{-x}}
\end{equation}
For each step in time, $t$, with $t<\tau_i$, clj-ctrnn will
\begin{enumerate}
\item Approximate the rate of membrane potential change, $\frac{\Delta y}{\Delta t}$, for all neurons
\item Synchronously update all membrane potentials $y_{i+1} = y_i + \frac{\Delta y}{\Delta t} \cdot t$
\item Commit the new network state
\end{enumerate}

\end{document}
